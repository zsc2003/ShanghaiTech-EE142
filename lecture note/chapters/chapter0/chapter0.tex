\chapter{Summary}


\begin{tikzpicture}[
    every node/.style={rectangle, draw, align=center},
    arrow/.style={-Stealth}
]
\centering
\footnotesize
\node (AEP) {Asymptotic \\ Equipartition \\ Property};
\node (discrete) [above left=15pt and 30pt of AEP] {Discrete Entropy, \\ Mutual Information};
\node (sourceencoding) [left=52pt of AEP] {Source \\ Encoding};
\node (channelencoding) [below left=15pt and 52pt of AEP] {Channel \\ Encoding};

\node (continuous) [above right=15pt and 30pt of AEP] {Continuous Entropy, \\ Mutual Information};
\node (quantatization) [right=43pt of AEP] {Quantatization};
\node (Gaussianchannel) [below right=15pt and 57pt of AEP] {Gaussian \\ Channel};

\draw [arrow] (discrete) -- (AEP);
\draw [arrow] (continuous) -- (AEP);
\draw [arrow] (AEP) -- (channelencoding);
\draw [arrow] (AEP) -- (sourceencoding);
\draw [arrow] (AEP) -- (Gaussianchannel);
\draw [arrow] (AEP) -- (quantatization);
\draw [arrow] (discrete) -- (sourceencoding);
\draw [arrow] (sourceencoding) -- (channelencoding);
\draw [arrow] (continuous) -- (quantatization);
\draw [arrow] (quantatization) -- (Gaussianchannel);
\end{tikzpicture}

A.E.P. 将所有的章节连接起来.

主要围绕香农三大定理展开: (注意成立的条件! 所有事情要满足其条件才行!)
\begin{enumerate}
\item Lossless Source Coding Theorem (\textcolor{red}{前提: $x^n\stackrel{i.i.d.}{\sim} p(x)$})
$$L^* = H(X)$$

\item Channel Coding Theorem (\textcolor{red}{前提: Memoryless Channel})
$$C^* = \max_{p(x)}I(X;Y)$$

\item Rate Distortion Theorem (\textcolor{red}{前提: $x^n\stackrel{i.i.d.}{\sim} p(x)$})
$$R(D) = \min_{p(x), \mathbb{E}(X)\leq p^2}I(X;\hat{X})$$
\end{enumerate}

Some interesting topics but not covered in class:
$\S 11.3$ Universal Source Coding, $\S 11.10$ Fisher Information, $\S 15.10$ General Multiterminal Networks.

$\ldots\ldots\ldots\ldots$ A lot of interesting topics $\ldots\ldots\ldots\ldots$